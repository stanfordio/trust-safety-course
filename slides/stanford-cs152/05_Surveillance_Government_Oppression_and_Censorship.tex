%! TEX program = LuaTeX

\documentclass[nobackground,dvipsnames,table,aspectratio=169]{beamer}
\usepackage{cs152}

\mode<presentation>
{\usetheme{Hannover}
    \usecolortheme{cs152}
    \setbeamercovered{transparent}
    \useinnertheme[shadow=false]{rounded}
    \usebackgroundtemplate{}
    \setbeamercolor*{frametitle}{parent=palette primary}
    \setbeamerfont{block title}{size={}}
    \setbeamertemplate{navigation symbols}{}
}

\title{Surveillance and Censorship}
\subtitle{CS 152 --- Trust and Safety Engineering}

\author[A. Stamos]{Alex Stamos}
\institute[Stanford University]{Stanford Cyber Policy Center}
\date[2022]{\today}
\subject{CS 152 --- Trust and Safety Engineering}
%\titlegraphic{\includegraphics[width=5cm]{img/cyber-logo-white-black-red-WEB}}

% Change the level of bulleting on the ToC page
\setcounter{tocdepth}{2}

\graphicspath{{img/lesson05}}

%TODO2 CHECK ORDER OF SLIDES WITH PROFESSOR STAMOS
%Most other todos are formatting or image/video embedding related

\begin{document}

\begin{frame}
    \titlepage
\end{frame}

\begin{frame}{Ahmed Mansoor}
    \includegraphics[width=\textwidth]{free-ahmed-mansoor}
    \textit{International Service for Human Rights}
\end{frame}

\begin{frame}{The UAE Had Help From Democractic Countries...}
    \includegraphics[width=\textwidth]{uae-help}
\end{frame}

\begin{frame}{Mansoor Remains Imprisoned in Solitary Confinement}
    \centering
    \includegraphics[width=\textwidth]{mansoor-health-at-risk}
    \textit{“The UAE’s persecution of Ahmed Mansoor for his ‘thought crimes’ has been cruel, even gratuitous. International sporting and cultural events...cannot mask the terrible spectacle of the country’s leading rights defender alone in a bare cell in inhumane conditions.”}\\
    \small{\textit{- Human Rights Watch, 2020}}
\end{frame}

\begin{frame}{What Will We Learn Today?}
    \begin{itemize}
        \item The difference between the idealized version of the internet and reality of its physical implementation
        \item The ways governments get access to data
        \item How internet censorship works
        \item How these techniques can sometimes be used by domestic abusers
    \end{itemize}
\end{frame}

\section{The Internet: Myth vs. Reality}

\begin{frame}{The Internet Has Always Had an Anarchist Streak}
    \begin{columns}[c]
        \column{0.2\textwidth}
            \includegraphics[width=\textwidth]{time-castro}
        \column{0.8\textwidth}
            \scriptsize
            That nation is about to get even bigger as the major commercial computer networks - Prodigy, CompuServe, America Online, GEnie and Delphi Internet Service - begin to dismantle the walls that have separated their private operations from the public Internet. The success of the Internet is a matter of frustration to the owners of the commercial networks, who have tried all sorts of marketing tricks and still count fewer than 5 million subscribers among them. Most commercial networks now allow electronic mail to pass between their services and the Internet. Delphi, which was purchased by Rupert Murdoch's News Corp. in September, began providing its customers full Internet access earlier this year. America Online (which publishes an electronic version of TIME) is scheduled to begin offering limited Internet services later this month.\\~\\
            
            People who uses these new entry points into the Net may be in for a shock. Unlike the family-oriented commercial services, which censor messages they find offensive, the Internet imposes no restrictions. Anybody can start a discussion on any topic and say anything. There have been sporadic attempts by local network managers to crack down on the raunchier discussion groups, but as Internet pioneer John Gilmore puts it, \textbf{"The Net interprets censorship as damage and routes around it."}\\~\\
            
            -Time Magazine, “First Nation in Cyberspace”, 6DEC1993
    \end{columns}
\end{frame}

\begin{frame}{"If you want to liberate a society, all you need is the Internet." - \textbf{Wael Ghonim}}
    \includegraphics[width=\textwidth]{protest}
\end{frame}

\begin{frame}{Technology to Organize and Protest?}
    \includegraphics[width=\textwidth]{percent-using-internet}
\end{frame}

\begin{frame}{The Anarchist Ideal Conflicts with the Centralized Reality}
    \includegraphics[width=\textwidth]{internet-cable-map}
\end{frame}

\begin{frame}{“I once said, if you want to liberate a society, all you need is the Internet. I was wrong.” - Wael Ghonim}
    \begin{columns}
        \column{0.5\textwidth}
            \includegraphics[width=\textwidth]{egypt-tracking-citizens}
        \column{0.5\textwidth}
            \includegraphics[width=\textwidth]{egypt-tracking-evidence}
            \href{https://research.checkpoint.com/2019/the-eye-on-the-nile/}{Check Point Research}
    \end{columns}
\end{frame}

\begin{frame}{Terminology}
    \begin{columns}[T]
        \column{0.9\textwidth}
            Surveillance\\
            \small{\textit{The act of observing persons or groups either with notice or their knowledge (overt surveillance) or without their knowledge (covert surveillance).}}\\~\\
        \column{0.1\textwidth}
            \includegraphics[width=\textwidth]{eyes}
    \end{columns}
    \vspace{0.05\textheight}
    \begin{columns}[T]
        \column{0.9\textwidth}            
            Censorship\\
            \small{\textit{The suppression of words, images, or ideas that are “offensive” whenever some succeed in imposing their personal political or moral values on others. Censorship by the government is unconstitutional.}}\\~\\
        
            \scriptsize
            \href{https://www.law.cornell.edu/wex/surveillance}{Cornell Law, }\href{https://www.aclu.org/other/what-censorship}{ACLU}
        \column{0.1\textwidth}
            \includegraphics[width=\textwidth]{zipped}
    \end{columns}
\end{frame}

\begin{frame}{Categories of Government Surveillance}
    \begin{itemize}
        \item Legal requests to providers
        \item Targeted attacks
        \item Mass sniffing
    \end{itemize}
\end{frame}

\section{Legal Requests}

\begin{frame}{}
    \thispagestyle{empty}
    \AddToShipoutPictureBG*{\includegraphics[width=\paperwidth]{4th-amendment}}%TODO3 alignment
\end{frame}

\begin{frame}{The Fourth Amendment}
    “The right of the people to be secure in their  persons,  houses,  papers,  and effects,  against  unreasonable  searches and  seizures,  shall  not  be  violated,  and no  Warrants  shall  issue,  but  upon probable  cause,  supported  by  Oath  or affirmation,  and  particularly  describing the  place  to  be  searched,  and  the persons or things to be seized.”\\
    \small{The United States Bill of Rights}
\end{frame}

\begin{frame}{Key US Laws}
    \begin{itemize}
        \item In 1968, after \textit{Berger} and \textit{Katz}, Congress passed original Wiretap Act
        \begin{itemize}
            \item Enacted as Title III of the Omnibus Crime Control and Safe Streets Act of 1968
        \end{itemize}
        
        \item In 1986, Congress passed ECPA
        \begin{itemize}
            \item Amended the Wiretap Act (now Title I of ECPA) -- extended it to include transmissions of electronic data via computers. 18 U.S.C. § 2510 \textit{et seq.}
            \item Added the Stored Communications Act (Title II of ECPA) -- applies to stored electronic communications. 18 U.S.C. § 2701 \textit{et seq.}
            \item Added the Pen Register Statute (Title III of ECPA). 18 U.S.C. § 3121 \textit{et seq.}
            \begin{wrapfigure}{L}{0.2\textwidth}
                \includegraphics[width=0.18\textwidth]{old-computer}
            \end{wrapfigure} %TODO3 get it to wrap around bullet points?
        \end{itemize}
        
        \item Remember: 4A is a \textit{floor}; Congress can and does pass laws that provide \textit{more} protection than the courts say the 4A provides.
    \end{itemize}
\end{frame}

\begin{frame}{ECPA Definitions}
    \textbf{Contents: "substance, purport, or meaning"} of the communication
    \begin{itemize}
        \item What the communication means, conveys, says
        \item Example: subject line + body of an email (or contents of a snail-mail letter)\\~\\
    \end{itemize}

    \textbf{Non-content} information = information \textbf{about} the communication
    \begin{itemize}
        \item "Metadata"
        \item Think: header of an email (or the outside of the envelope the letter is sealed in)\\~\\
    \end{itemize}
    
    Content/non-content distinction becomes really blurry in the context of the Internet
    \begin{itemize}
        \item Internet's complex architecture means one particular unit of data might change status (from content to non-content, or vice versa) as it travels from sender to recipient
        \item Content or non-content status might also depend on where in the network that unit of data resides at that particular moment in time
        \item We no longer live in the world of \textit{Katz} and \textit{Smight} -- hence calls for reforming ECPA
    \end{itemize}
\end{frame}

\begin{frame}{SCA: Who + What + How}
    \includegraphics[width=\textwidth]{sca-table}
    \tiny
    Source: adapted from Perkins Cole
\end{frame}

\begin{frame}{}
    \thispagestyle{empty}
    \AddToShipoutPictureBG*{\includegraphics[height=\paperheight]{exonerating-evidence}}%TODO3 image alignment
\end{frame}

\begin{frame}{FISA and FAA}
    \includegraphics[width=\textwidth]{nixon-resignation}
\end{frame}

\begin{frame}{FISA and FAA}
    \includegraphics[width=\textwidth]{section-702-flowchart}
    \tiny
    \url{https://www.thirdway.org/report/guide-to-section-702-reform}
\end{frame}

\begin{frame}{How About Outside of the US?}
    %TODO2 government data requests graph - there are two of them for some reason???
    \tiny
    \url{https://transparency.facebook.com/government-data-requests}
\end{frame}

\section{Targeted Surveillance}

\begin{frame}{A Controversial TOR Takedown: Playpen}
    \begin{columns}
        \column{0.4\textwidth}
            \includegraphics[width=1.5\textwidth]{playpen-article}
        \column{0.6\textwidth}
            \includegraphics[width=\textwidth]{playpen-court-article}
    \end{columns}
\end{frame}

\begin{frame}{Commercial Hacking}
    \begin{columns}
        \column{0.7\textwidth}
            \includegraphics[width=\textwidth]{pss-commercial-hacking}
        \column{0.3\textwidth}
            \includegraphics[width=\textwidth]{testbot1}
    \end{columns}
    \tiny
    \url{https://citizenlab.ca/2017/12/champing-cyberbit-ethiopian-dissidents-targeted-commercial-spyware/}
\end{frame}

\begin{frame}{Surveillance and Malware Meet: NSO Group}
    \includegraphics[width=\textwidth]{pegasus-infection-map}
\end{frame}

\begin{frame}{WhatsApp vs. NSO Group}
    \begin{columns}
        \column{0.45\textwidth}
            \includegraphics[width=\textwidth]{whatsapp-vs-nso-court-doc}
        \column{0.55\textwidth}
            \includegraphics[width=\textwidth]{nso-rant}
    \end{columns}
\end{frame}

\begin{frame}{Anybody Can Be a Target}
    \begin{columns}
        \column{0.7\textwidth}
            \includegraphics[width=\textwidth]{project-cato-executive-summary}
        \column{0.3\textwidth}
            \includegraphics[width=\textwidth]{project-cato-photos}
    \end{columns}
\end{frame}

\section{Mass Surveillance}

\begin{frame}{}
    \thispagestyle{empty}
    \AddToShipoutPictureBG*{\includegraphics[width=\paperwidth]{prism1}}
\end{frame}

\begin{frame}{}
    \thispagestyle{empty}
    \AddToShipoutPictureBG*{\includegraphics[width=\paperwidth]{prism2}}
\end{frame}

\begin{frame}{}
    \thispagestyle{empty}
    \AddToShipoutPictureBG*{\includegraphics[width=\paperwidth]{room-641a}}
\end{frame}

\begin{frame}{}
    \thispagestyle{empty}
    \AddToShipoutPictureBG*{\includegraphics[width=\paperwidth]{prism-yahoo-webmessenger}}
\end{frame}

\section{Censorship}

\begin{frame}{There Are Also Countries With Excellent Blocking Capabilities}
    \includegraphics[width=\textwidth]{china-blocking-diagram}
\end{frame}

\begin{frame}{One of the Solutions Has Been the VPN}
    \includegraphics[width=\textwidth]{vpn-solution}
\end{frame}

\begin{frame}{Enter The Onion Router}
    \includegraphics[width=\textwidth]{tor}
\end{frame}

\begin{frame}{How Does Tor Work?}
    \includegraphics[width=\textwidth]{how-tor-works}
\end{frame}

\begin{frame}{Weaknesses in Tor}
    \includegraphics[width=\textwidth]{tor-weaknesses}
\end{frame}

\begin{frame}{TOR Is in an Arms Race with the PRC}
    \includegraphics[width=\textwidth]{china-firewall-race}
    \tiny{\href{https://blog.thousandeyes.com/the-war-between-chinas-great-firewall-and-circumvention-tools/}{blog.thousandeyes.com/the-war-between-chinas-great-firewall-and-circumvention-tools/}}
\end{frame}

\begin{frame}{PRC Detection of Tor}
    \includegraphics[width=\textwidth]{prc-tor-detection}
\end{frame}

\begin{frame}{.onion Services}%TODO2 something seems to be missing/arrows seem to be wrong
    \includegraphics[width=\textwidth]{onion-services}
\end{frame}

\begin{frame}{The Site That Made the Dark Web Famous:}
    \includegraphics[width=\textwidth]{silkroad}
\end{frame}

\begin{frame}{The Site That Made the Dark Web Famous:}
    \includegraphics[width=\textwidth]{silkroad-seized}
\end{frame}

\begin{frame}{How Do Countries Block?}
    \begin{columns}
        \column{0.5\textwidth}
            \begin{enumerate}
                \item By DNS name
                \begin{enumerate}
                    \item \small{Defeated by alternative DNS, VPN}
                \end{enumerate}
        
                \item By IP address
                \begin{enumerate}
                    \item \small{Defeated by domain fronting, cloud CDNs, VPN}
                \end{enumerate}
        
                \item By SNI sniffing
                \begin{enumerate}
                    \item \small{Addressed by TLS 1.3, VPN}
                \end{enumerate}
        
                \item Full Internet shutdowns
            \end{enumerate}
        \column{0.5\textwidth}
            \includegraphics[width=\textwidth]{iran-internet-shutdown-2019}
    \end{columns}
\end{frame}

\begin{frame}{Internet Shutdowns Are on the Rise}
    \small{In 2019, the \#KeepItOn Coalition recorded \textbf{213 incidents} of Internet Shutdowns across \textbf{33 countries.}}
    \begin{columns}
        \column{0.5\textwidth}
            \includegraphics[width=\textwidth]{unganda-internet-shutdown}
            \includegraphics[width=\textwidth]{guinea-internet-disruptions}
        \column{0.5\textwidth}
            \includegraphics[width=\textwidth]{shutdown-justifications}
    \end{columns}
\end{frame}

\begin{frame}{Surveillance/Censorship Capabilities by Country Category}
    \footnotesize %TODO3 table formatting
    \begin{tabularx}{\textwidth}{|X|X|X|X|}
        \hline
        \textbf{Country Category} & \textbf{Surveillance Capabilities} & \textbf{Censorship Capabilities} & \textbf{Possible Constraints}\\
        Highly-capable democracies & Custom malware Tapping of internet backbones Voluntary disclosure of corporate data Compelled disclosure of corporate data Intelligence sharing & Expansion of private power->control by big companies such as Facebook and Twitter. Not illegal, but problematic. & Legal action Administrative procedures\\
        \hline
        \hline
        Highly-capable autocracies & Custom malware Quasi-state hackers Compelled disclosure of corporate data (sometimes) Voluntary disclosure of corporate data (rarely) Insider data access abuse & Deep-packet inspection ML-based blocking inside of platforms & Limits on data collection Export controls\\
        \hline
        Less-capable states & Commercial spyware Arrest / compelled disclosure of personal data Voluntary disclosure of corporate data (sometimes) & Social media blocking (DNS/IP) Internet throttling/shutdowns & Malware detection Export controls\\
        \hline
    \end{tabularx}
\end{frame}

\begin{frame}{Domestic Surveillance}
    \centering
    \textit{Or where we start making you paranoid….}
    \begin{figure}
        \centering
        \includegraphics[height=0.6\textheight]{americans-poster}
    \end{figure}
\end{frame}

\begin{frame}{Domestic Abusers Are the Hardest Adversary}
    \begin{columns}
        \column{0.6\textwidth}
            \begin{enumerate}
                \item Often unlimited physical access to devices
                \begin{enumerate}
                    \item Phones
                    \item Computers
                    \item Household IoT
                \end{enumerate}
                \item Shared passwords and emails
                \item Integrated into the victim’s social network
                \item Knowledge of all password reset questions
                \item Other forms of legal/financial leverage
            \end{enumerate}
        \column{0.4\textwidth}
            \includegraphics[width=\textwidth]{security-cameras}
    \end{columns}
\end{frame}

\begin{frame}{Possible Response: The "Break Up" Flow}
    \centering
    \includegraphics[height=\textheight]{breakup-flow}
\end{frame}

\begin{frame}{"Legitimate" Spyware}
    \begin{columns}
        \column{0.5\textwidth}
            OK Use:
            \begin{itemize}
                \item Transparently installing apps on your child’s phone so you know where they are
                \item Installing software to find your device if it is lost or stolen
                \item Consensual on both sides
            \end{itemize}
        \column{0.5\textwidth}
            Not OK Use:
            \begin{itemize}
                \item Covertly installing apps on your intimate partner’s phone so you know where they are
                \item Exploiting shared passwords to view private communications
            \end{itemize}
    \end{columns}
    \vspace{0.1\textheight}
    \begin{columns}
        \column{0.3\textwidth}
            \includegraphics[width=\textwidth]{kidguard}
        \column{0.2\textwidth}
            \includegraphics[width=\textwidth]{find-my-phone}
        \column{0.3\textwidth}
            \includegraphics[width=\textwidth]{spyzie}
        \column{0.2\textwidth}
            \includegraphics[width=\textwidth]{mspy}
    \end{columns}
\end{frame}

\begin{frame}{Smart Home: Convenience or Abuse?}
    \includegraphics[width=\textwidth]{smart-home}
\end{frame}

\begin{frame}{Apple Made Stalking Tools Pretty With a Better UX}
    \begin{columns}
        \column{0.45\textwidth}
            \includegraphics[width=\textwidth]{airtags-nyt-headline}
            \includegraphics[width=\textwidth]{airtags}
        \column{0.55\textwidth}
            \includegraphics[width=\textwidth]{airtags-nyt-article}
    \end{columns}
\end{frame}

\section{Responses and Mitigations}

\begin{frame}{Policy Responses}
    \begin{enumerate}
        \item Minimize data collection.
        \item Create clear, human-rights based policies on government data access.
        \item Publish Transparency Reports on Government Data Access.
        \item Protect users especially vulnerable to surveillance.
        \item Provide digital security training and resources for vulnerable groups.
        \item Take a stand against stalkerware.
        \item Restrict employee access to private data.
    \end{enumerate}
\end{frame}

\begin{frame}{Public: Wassenaar and Other Attempts to Regulate Spyware}
    \textbf{1996 Wassennaar Arrangement} - the first global multilateral arrangement on export controls for conventional weapons and sensitive dual-use goods and technologies. \textbf{Revised in 2013 to include internet-based surveillance systems.}\\~\\
    \footnotesize
    \href{https://www.wassenaar.org/the-wassenaar-arrangement/}{wassenaar.org}
    \begin{columns}
        \column{0.5\textwidth}
            \includegraphics[width=\textwidth]{wassenaar-overview}
        \column{0.5\textwidth}
            \includegraphics[width=\textwidth]{wassenaar-signatories}
    \end{columns}
\end{frame}

\begin{frame}{Restricts Sale and Distribution of Surveillance Tools to Oppressive Regimes}
    \includegraphics[width=\textwidth]{wassenaar-licensing-factors}
\end{frame}

\begin{frame}{Unintended Consequences of the Wassenaar Arrangement}
    \includegraphics[width=\textwidth]{wired-arms-control}
    \begin{itemize}
        \item Rules so broad could criminalize some research, bug bounty programs, and penetration testing and restrict legitimate security tools 
        \item Some argued the restrictions weakened security of participating nations and did little to curb threats from non-participant nations
        \item Also hard to enforce 
    \end{itemize}
\end{frame}

\begin{frame}{Al Jazeera Investigation Shows Spyware Firms Breach Agreements}
    %TODO2 embed video link so the time limit is transferred
\end{frame}

\begin{frame}{Private Action: WhatsApp vs. NSO }
    \includegraphics[width=\textwidth]{nso-whatsapp}
\end{frame}

\begin{frame}{Product/Technical Responses}
    \begin{enumerate}
        \item Expand the use of encryption to protect user data
        \item Build in Malware Detection Mechanisms
        \item Build Account Takeover Detection Mechanisms
        \item Consider the needs of abused individuals in your product design
    \end{enumerate}
\end{frame}

\begin{frame}{Government Attacks on E2EE}
    \begin{columns}
        \column{0.45\textwidth}
            \includegraphics[width=\textwidth]{earn-it-wired}
        \column{0.55\textwidth}
            \includegraphics[width=\textwidth]{encryption-meeting-follow-up}
    \end{columns}
\end{frame}

\begin{frame}{Where Is This Going?}
    \begin{itemize}
        \item Big-country capabilities are trickling down/being sold to many nations
        \item Targeted malware has become normalized  against journalists /activists
        \item Recent success against stalkerware not sufficient 
        \item New embedded tech makes domestic spying very easy
    \end{itemize}
\end{frame}

\end{document}